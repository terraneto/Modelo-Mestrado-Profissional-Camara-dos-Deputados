% ------------------------------------------------
% ------------------------------------------------
% abnTeX2: Modelo de Relatório Técnico/Acadêmico em conformidade com 
% ABNT NBR 10719:2015 Informação e documentação - Relatório técnico e/ou
% científico - Apresentação
% ----------------------------------------------- 
% -----------------------------------------------
\documentclass[
	% -- opções da classe memoir --
	12pt,				% tamanho da fonte
	openright,			% capítulos começam em pág ímpar (insere página vazia caso preciso)
	twoside,			% para impressão em recto e verso. Oposto a oneside
	a4paper,			% tamanho do papel. 
	% -- opções da classe abntex2 --
	chapter=TITLE,		% títulos de capítulos convertidos em letras maiúsculas
	%section=TITLE,		% títulos de seções convertidos em letras maiúsculas
	%subsection=TITLE,	% títulos de subseções convertidos em letras maiúsculas
	%subsubsection=TITLE,% títulos de subsubseções convertidos em letras maiúsculas
	% -- opções do pacote babel --
	english,			% idioma adicional para hifenização
	french,				% idioma adicional para hifenização
	spanish,			% idioma adicional para hifenização
	brazil,				% o último idioma é o principal do documento
	%sumario=tradicional, % Sumário identado
	]{abntex2}
	
% ---
% PACOTES
% ---

% ---
% Pacotes fundamentais 
% ---
\usepackage{lmodern}			% Usa a fonte Latin Modern
\usepackage[T1]{fontenc}		% Selecao de codigos de fonte.
\usepackage[utf8]{inputenc}		% Codificacao do documento (conversão automática dos acentos)
\usepackage{indentfirst}		% Indenta o primeiro parágrafo de cada seção.
\usepackage{color}				% Controle das cores
\usepackage{graphicx}			% Inclusão de gráficos
\usepackage{microtype} 			% para melhorias de justificação
\usepackage{pdfpages}
\usepackage[makeindex]{imakeidx} % Índice remissivo

% ---

% ---
% Pacotes adicionais, usados apenas no âmbito do Modelo Canônico do abnteX2
% ---
\usepackage{lipsum}				% para geração de dummy text
% ---

% ---
% Pacotes adicionais, usados no anexo do modelo de folha de identificação e tabelas e quadros longos
% ---
\usepackage{multicol}
\usepackage{multirow}
\usepackage{longtable,ltcaption}
% ---
	
% ---
% Pacotes de glossário
% ---
\let\printglossary\relax
\let\theglossary\relax
\let\endtheglossary\relax
\usepackage[style=altlist,nonumberlist]{glossaries}
\newglossaryentry{overfitting}{
    name={Overfitting},
    text={overfitting},
    description={\textit{Overfitting}, Sobre-ajuste ou Sobreajuste é um termo usado em estatística para descrever quando um modelo estatístico se ajusta muito bem ao conjunto de dados anteriormente observado, mas se mostra ineficaz para prever novos resultados.
    }
}


% ---
% Exemplo de configurações do glossário
\renewcommand*{\glsseeformat}[3][\seename]{\textit{#1}  
 \glsseelist{#2}}
% ---

\makeglossaries

% ---
% Pacotes de citações
% ---
\usepackage[brazilian,hyperpageref]{backref}  % Paginas com as citações na bibl
\usepackage[alf,abnt-emphasize = bf,abnt-url-package=url]{abntex2cite}	% Citações padrão ABNT
\def\UrlLeft{}
\def\UrlRight{}

%----
% Pacote lista de Abreviações e siglas
%---
\usepackage{nomencl}
\makenomenclature
\renewcommand{\nomname}{Lista de Abreviaturas e Siglas}

% --- 
% CONFIGURAÇÕES DE PACOTES
% --- 

% ---
% Configurações do pacote backref
% Usado sem a opção hyperpageref de backref
\renewcommand{\backrefpagesname}{Citado na(s) página(s):~}
% Texto padrão antes do número das páginas
\renewcommand{\backref}{}
% Define os textos da citação
\renewcommand*{\backrefalt}[4]{
	\ifcase #1 %
		Nenhuma citação no texto.%
	\or
		Citado na página #2.%
	\else
		Citado #1 vezes nas páginas #2.%
	\fi}%
% ---

% Cria o comando \subtitulo. A norma define que o TITULO deve ser em caixa alta negrito, mas o subtitulo deve ser em caixa baixa. Outros comandos foram criados para preencher a folha de aprovação
\providecommand{\imprimirsubtitulo}{}
\newcommand{\subtitulo}[1]{\renewcommand{\imprimirsubtitulo}{#1}}

\providecommand{\imprimircurso}{}
\newcommand{\curso}[1]{\renewcommand{\imprimircurso}{#1}}

\providecommand{\imprimirsubinstituicao}{}
\newcommand{\subinstituicao}[1]{\renewcommand{\imprimirsubinstituicao}{#1}}

\providecommand{\imprimirlinhadepesquisa}{}
\newcommand{\linhadepesquisa}[1]{\renewcommand{\imprimirlinhadepesquisa}{#1}}

\providecommand{\imprimirmembroA}{}
\newcommand{\membroA}[1]{\renewcommand{\imprimirmembroA}{#1}}

\providecommand{\imprimirfuncaomembroA}{}
\newcommand{\funcaomembroA}[1]{\renewcommand{\imprimirfuncaomembroA}{#1}}

\providecommand{\imprimirmembroB}{}
\newcommand{\membroB}[1]{\renewcommand{\imprimirmembroB}{#1}}

\providecommand{\imprimirfuncaomembroB}{}
\newcommand{\funcaomembroB}[1]{\renewcommand{\imprimirfuncaomembroB}{#1}}

\providecommand{\imprimirdataaprovacao}{}
\newcommand{\dataaprovacao}[1]{\renewcommand{\imprimirdataaprovacao}{#1}}

% ---
% Informações de dados para CAPA e FOLHA DE ROSTO e Folha de aprovação
% ---
% ---
% Informações de dados para CAPA e FOLHA DE ROSTO e Folha de aprovação
% ---
\titulo{Modelo Canônico de TCC com \abnTeX:}
\subtitulo{Câmara dos Deputados}
\autor{Rubens Vasconcellos Terra Neto}
\local{Brasília}
\data{2022}
\orientador{Prof. Dr. XXXXX XXXXX XXXXX} 
\membroA{Prof. Dr. XXXXXX XXXXXX XXXXX}
\funcaomembroA{Membro - Câmara do Deputados}
\membroB{Prof. Dr. XXXXXX XXXXXX XXXXX}
\funcaomembroB{Membro - Externo}
\dataaprovacao{15 de junho de 2023}
\instituicao{Câmara dos Deputados}
\subinstituicao{Centro de Formação, Treinamento e Aperfeiçoamento}
\linhadepesquisa{Gestão Pública no Poder Legislativo}
\tipotrabalho{Tese (Mestrado)}
\curso{Mestrado Profissional em Poder Legislativo}
% O preambulo deve conter o tipo do trabalho, o objetivo,
% o nome da instituição e a área de concentração
\preambulo{Modelo canônico de Projeto de pesquisa em conformidade
com as normas ABNT apresentado à comunidade de usuários da Câmara dos Deputados.}
% ---



% ---
% Configurações de aparência do PDF final


% alterando o aspecto da cor azul
\definecolor{blue}{RGB}{41,5,195}

% informações do PDF
\makeatletter
\hypersetup{
     	%pagebackref=true,
		pdftitle={\@title}, 
		pdfauthor={\@author},
    	pdfsubject={\imprimirpreambulo},
	    pdfcreator={LaTeX with abnTeX2},
		pdfkeywords={CEFOR}{ }{latex}{ }{abntex}{ }{abntex2}, 
		colorlinks=true,       		% false: boxed links; true: colored links
    	linkcolor=black,          	% color of internal links
    	citecolor=black,        		% color of links to bibliography
    	filecolor=black,      	% color of file links
		urlcolor=black,
		bookmarksdepth=4
}
\makeatother
\renewcommand{\cftsectionfont}{\color{black}\normalfont\uppercase}
% --- 

% ---
% Possibilita criação de Quadros e Lista de quadros.
% Ver https://github.com/abntex/abntex2/issues/176
%
\newcommand{\quadroname}{Quadro}
\newcommand{\listofquadrosname}{Lista de quadros}

\newfloat[chapter]{quadro}{loq}{\quadroname}
\newlistof{listofquadros}{loq}{\listofquadrosname}
\newlistentry{quadro}{loq}{0}

% configurações para atender às regras da ABNT
\setfloatadjustment{quadro}{\centering}
\counterwithout{quadro}{chapter}
\renewcommand{\cftquadroname}{\quadroname\space} 
\renewcommand*{\cftquadroaftersnum}{\hfill--\hfill}

\setfloatlocations{quadro}{hbtp} % Ver https://github.com/abntex/abntex2/issues/176
% ---

% --- 
% Espaçamentos entre linhas e parágrafos 
% --- 

% O tamanho do parágrafo é dado por:
\setlength{\parindent}{1.3cm}

% Controle do espaçamento entre um parágrafo e outro:
\setlength{\parskip}{0.2cm}  % tente também \onelineskip

% ---
% compila o indice
% ---
\makeindex
% ---

%-----------------------------------------------------------------------
% REDEFINIÇÃO DA CAPA e Folha de Rosto
%-----------------------------------------------------------------------
%---
% REDEFINIÇÃO DA CAPA
%---
\renewcommand{\imprimircapa}{
\begin{capa}
    \center
    \includegraphics[scale=0.6]{imagens/brasao.jpg}
    
    \MakeUppercase{{\ABNTEXchapterfont\bfseries\imprimirinstituicao \\ \imprimirsubinstituicao \\ Programa de Pós-graduação \\ Mestrado Profissional em Poder Legislativo}}

    \vspace*{\fill}

    {\textsc{\ABNTEXchapterfont\bfseries\large\imprimirautor}}

    \vspace*{\fill}

    \MakeUppercase{{\ABNTEXchapterfont\bfseries\imprimirtitulo}}
    
    {\ABNTEXchapterfont\imprimirsubtitulo}

    \vspace*{\fill}

    \vspace*{\fill}

    {\large\imprimirlocal}
    \par
    {\large\imprimirdata}
    \vspace{1cm}
\end{capa}
}

%-------------------------
% Folha de Rosto
%---------------------------
\makeatletter
\renewcommand{\folhaderostocontent}{
    \begin{center}
        \par\textsc{{\ABNTEXchapterfont\bfseries\large\imprimirautor}}
        
        \vspace*{\fill}
        
        \vspace*{\fill}
        
        {\ABNTEXchapterfont\bfseries\Large\imprimirtitulo}
        
        {\ABNTEXchapterfont\Large\imprimirsubtitulo}
      
        \vspace*{\fill}
        
      \begin{flushright}
      \begin{minipage}{.5\textwidth}
      	 \SingleSpacing
         \imprimirpreambulo
         \par\bigskip
      \end{minipage}
      
      \end{flushright}
      \vspace*{\fill}
      \begin{flushleft}
      \begin{minipage}{.65\textwidth}
      	 \SingleSpacing
         \textbf{{\imprimirorientadorRotulo~\imprimirorientador}}
         Área de Concentração: Poder Legislativo
         
         Linha de Pesquisa: \imprimirlinhadepesquisa 

      \end{minipage}
      
      \end{flushleft}
      
       \vspace*{\fill}
    
        {\large\imprimirlocal}
        \par
        {\large\imprimirdata}
        \vspace{1cm}
    \end{center} 
}
\makeatother
%---------------------------------------------------------------

% ----
% Início do documento
% ----
\begin{document}

% Seleciona o idioma do documento (conforme pacotes do babel)
%\selectlanguage{english}
\selectlanguage{brazil}

% Retira espaço extra obsoleto entre as frases.
\frenchspacing 

% ------------------------------------------------------
% ELEMENTOS PRÉ-TEXTUAIS
% ------------------------------------------------------
% \pretextual

% ---
% Capa
% ---
\imprimircapa 
% ---

% ---
% Folha de rosto
% (o * indica que haverá a ficha bibliográfica)
% ---
\imprimirfolhaderosto*
% ---

% ---
% Anverso da folha de rosto:
% ---
% Folha de autorização
	\sffamily
%	\vspace*{\fill}					% Posição vertical
	\begin{center}					% Minipage Centralizado
	\fbox{\begin{minipage}[c][7cm]{13cm}
	Autorização
   
        \vspace{0.7cm}	
Autorizo a divulgação do texto completo no sítio da Câmara dos Deputados e a reprodução total ou parcial, exclusivamente, para fins acadêmicos e científicos.

        \vspace{0.7cm}
        
Assinatura: \_\_\_\_\_\_\_\_\_\_\_\_\_\_\_\_\_\_\_\_\_\_\_\_\_\_\_\_\_\_\_\_\_\_\_ 

        \vspace{0.7cm}
        
Data: \_\_\_/\_\_\_/\_\_\_

     \hspace{9cm}
	\end{minipage} }
	
	\end{center}
%--------
\begin{fichacatalografica}
	\sffamily
	\vspace*{\fill}					% Posição vertical
	\begin{center}					% Minipage Centralizado
	\fbox{\begin{minipage}[c][9cm]{13cm}
	       	\hspace{1.0cm}\begin{minipage}[c][9cm] {11cm}
	    \small
	    \hspace{1.0cm}Terra Neto, Rubens Vasconcellos.	
	    
	    \hspace{1.0cm} 

	     \hspace{0.5cm}\imprimirtitulo{ }\imprimirsubtitulo / \imprimirautor.{ }-- \imprimirdata.
	     \hspace{0.5cm}\thelastpage p. : il.\\


        \hspace{0.5cm}\imprimirorientadorRotulo~\imprimirorientador.

	    \hspace{1.0cm} 
	    
        \hspace{0.5cm}Impresso por computador.
        
        \hspace{0.5cm}"\imprimirpreambulo"


		    \hspace{0.5cm}1. Inteligência Artificial. 
		    2. Setor público, contratação.
		    3. Brasil. Congresso Nacional. Senado Federal.
		    4. Poder Legislativo
		    I. Título.
		    
	    \vspace{0.6cm} 
        \hspace{8.5cm}CDD 004.6782
        \end{minipage}
        \end{minipage}
}
	
        \vspace{0.6 cm}  
	
	\end{center}

\end{fichacatalografica}
% ---
% Inserir folha de aprovação
% ---
% ---
% Inserir folha de aprovação
% ---

% Isto é um exemplo de Folha de aprovação, elemento obrigatório da NBR
% 14724/2011 (seção 4.2.1.3). Você pode utilizar este modelo até a aprovação
% do trabalho. Após isso, substitua todo o conteúdo deste arquivo por uma
% imagem da página assinada pela banca com o comando abaixo:
%
%\begin{folhadeaprovacao}
%\includepdf{FolhadeAprovacao.pdf}
%\end{folhadeaprovacao}
%
%\begin{comment}
\begin{folhadeaprovacao}
\begin{flushleft}
 \begin{minipage}{3cm}
   \includegraphics[scale=0.57]{imagens/brasao.jpg}      
 \end{minipage} 
 \begin{minipage}{10cm}
    CÂMARA DOS DEPUTADOS\\Centro de Formação, Treinamento e Aperfeiçoamento\\
    Programa de Pós-Graduação\\Mestrado Profissional em Poder Legislativo
 \end{minipage} 
 \centerline{\rule{16.2cm}{0.4pt}}
\end{flushleft}
 
\center \textsc{\ABNTEXchapterfont\bfseries\large{FOLHA DE APROVAÇÃO}}
 
\vspace{0.5cm}

\begin{minipage}{15.9cm}

\textbf{Título:{ }}\textsc{\imprimirtitulo}{ }\textsc{\imprimirsubtitulo}

\textbf{Autor: } {\bfseries\imprimirautor}

\textbf{Área de concentração: }{Poder Legislativo}

\textbf{\textsc{Linha de Pesquisa: }} \imprimirlinhadepesquisa


\vspace{1.2cm}

Trabalho de conclusão de curso submetido à Comissão Examinadora designada pela Coordenação do Programa de Pós-graduação do Centro de Formação, Aperfeiçoamento e Treinamento da Câmara dos Deputados como requisito parcial para obtenção do título de \textbf{Mestre} em Poder Legislativo.

\vspace{1.2cm}

Trabalho aprovado em \imprimirdataaprovacao.

\begin{flushleft}
 
\vspace{1.5cm}
\_\_\_\_\_\_\_\_\_\_\_\_\_\_\_\_\_\_\_\_\_\_\_\_\_\_\_\_\_\_\_\_\_\_\_\\
\textbf{\imprimirorientador}\\
Presidente da Banca - Câmara dos Deputados

\vspace{1.5cm}
\_\_\_\_\_\_\_\_\_\_\_\_\_\_\_\_\_\_\_\_\_\_\_\_\_\_\_\_\_\_\_\_\_\_\_\\
\textbf{\imprimirmembroA}\\
\imprimirfuncaomembroA

\vspace{1.5cm}
\_\_\_\_\_\_\_\_\_\_\_\_\_\_\_\_\_\_\_\_\_\_\_\_\_\_\_\_\_\_\_\_\_\_\_\\
\textbf{\imprimirmembroB}\\
\imprimirfuncaomembroB

\vspace{1cm}

\end{flushleft}

\end{minipage}   

\end{folhadeaprovacao}
%\end{comment}

% ---

% ---
% Dedicatória
% ---
% ---
% Dedicatória
% ---
\begin{dedicatoria}
   \vspace*{\fill}
   \centering
   \noindent
   \textit{ Dedico este trabalho à minha família, que sempre me incentivou.} 
   \vspace*{\fill}
\end{dedicatoria}
% ---
% ---

% \include{0-PreTextual/Errata}

% ---
% Agradecimentos
% ---
% ---
% Agradecimentos
% ---
\begin{agradecimentos}
Agradeço ao meu orientador \imprimirorientador, pela sabedoria com que me guiou nesta trajetória.

Aos meus colegas de mestrado, que dividiram comigo a busca constante de conhecimento e aprimoramento profissional.

Aos professores do curso cuja dedicação foi fator preponderante para a qualidade do curso.

A todo o pessoal da COPOS que tanto se dedicaram por nós do corpo discente.

Gostaria de deixar registrado também, o meu reconhecimento à minha família, pois acredito que sem o apoio deles seria muito difícil vencer esse desafio.

Enfim, a todos os que por algum motivo contribuíram para a realização desta pesquisa.

\end{agradecimentos}
% ---
% ---

% ---
% Epígrafe
% ---
% ---
% Epígrafe
% ---
\begin{epigrafe}
    \vspace*{\fill}
    \begin{flushright}
	\begin{minipage}{.50\textwidth}
		\textit{``As máquinas de previsão não fornecem julgamentos. Apenas os humanos o fazem, porque apenas os humanos podem expressar as recompensas relativas de realizar ações diferentes.''}
		\begin{flushright}(Ajay Agrawal)    \end{flushright}
    \end{minipage}
    \end{flushright}
\end{epigrafe}
% ---

% ---

% ---
% RESUMOS
% ---
\include{0-PreTextual/Resumos}
% ---

% ---
% inserir lista de ilustrações
% ---
\pdfbookmark[0]{\listfigurename}{lof}
\listoffigures*
\cleardoublepage
% ---

% ---
% inserir lista de quadros
% ---
\pdfbookmark[0]{\listofquadrosname}{loq}
\listofquadros*
\cleardoublepage
% ---

% ---
% inserir lista de tabelas
% ---
\pdfbookmark[0]{\listtablename}{lot}
\listoftables*
\cleardoublepage
% ---

% ---
% inserir lista de abreviaturas e siglas
% ---
\printnomenclature[2cm] % entre colchetes é o espaçamento entre a sigla e seu significado
\clearpage
% A lista de abreviaturas e siglas está sendo gerada automaticamente, o que você precisa é incluir uma linha no seu texto que tenha o seguinte formato:
% \nomenclature {APF} {Administração Pública Federal}
% \nomenclature {sigla} {explicação}

% ---
% ---

% ---
% inserir lista de símbolos
% ---
%\include{0-PreTextual/ListadeSimbolos}
% ---

% ---
% inserir o sumario
% ---
\pdfbookmark[0]{\contentsname}{toc}
\tableofcontents*
\cleardoublepage
% ---

% ------------------------------------------------
% ELEMENTOS TEXTUAIS
% ------------------------------------------------
\textual

% ------------------------------------------------
% Introdução 
% ------------------------------------------------
\include{1-Introducao/introducao}

% ------------------------------------------------
% Fundamentação Teórica 
% ------------------------------------------------
% -------------------------------------------------------
% Fundamentação Teórica 
% -------------------------------------------------------
% ----------------------------------------------------------
% PARTE
% ----------------------------------------------------------
\part{Referenciais teóricos}
% ----------------------------------------------------------

% ---
% Capitulo de revisão de literatura
% ---
\chapter{Lorem ipsum dolor sit amet}
% ---

% ---
\section{Aliquam vestibulum fringilla lorem}
% ---

\lipsum[1]

\section{Teste de nova seção}

\lipsum[2-3]
\include{1-Introducao/abntex2-modelo-include-comandos}
% ------------------------------------------------
% Desenvolvimento
% ------------------------------------------------
%-------------------------------------------------------
% Parte Desenvolvimento
%-------------------------------------------------------
%\part{Desenvolvimento}

% -------------------------------------------------------
% Desenvolvimento
% -------------------------------------------------------
\chapter[Desenvolvimento]{Desenvolvimento}

\lipsum[31-33]

% ------------------------------------------------
% Resultados e Análises
% ------------------------------------------------
%\part{Resultados}
\include{2-Desenvolvimento/resultados}

% ---
% Finaliza a parte no bookmark do PDF
% para que se inicie o bookmark na raiz
% e adiciona espaço de parte no Sumário
% ---
\phantompart

% ------------------------------------------------
% Conclusão
% ------------------------------------------------
%\part{Conclusão}
\include{3-Conclusao/conclusao}
% ---

% ------------------------------------------------
% ELEMENTOS PÓS-TEXTUAIS
% ------------------------------------------------
\postextual

% ------------------------------------------------
% Referências bibliográficas
% Aqui está referência o arquivo .BIB que contém todas as referências
% ------------------------------------------------
\bibliography{Referencias/TCC-mestrado-referencias}

% ---
% Imprime o glossário
% ---
\cleardoublepage
\phantompart
\addcontentsline{toc}{chapter}{\glossaryname}
\printglossaries
% ---

% -----------------------------------------------
% Apêndices
% ------------------------------------------------
\include{Apendices&Anexos/Apendices}
% ---

% ------------------------------------------------
% Anexos
% ------------------------------------------------
\include{Apendices&Anexos/Anexos}

%-------------------------------------------------
% INDICE REMISSIVO
% lista as referências que se utilizou o \index
%-------------------------------------------------

\phantompart
\cleardoublepage
\addcontentsline{toc}{chapter}{ÍNDICE REMISSIVO}
\printindex

%-------------------------------------------------
% Formulário de Identificação (opcional)
%-------------------------------------------------
%\include{FormularioIdentificacao/formuláriodeidentificação}

\end{document}
